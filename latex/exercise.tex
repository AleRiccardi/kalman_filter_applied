\documentclass{article}

% If you're new to LaTeX, here's some short tutorials:
% https://www.overleaf.com/learn/latex/Learn_LaTeX_in_30_minutes
% https://en.wikibooks.org/wiki/LaTeX/Basics

% Formatting
\usepackage[utf8]{inputenc}
\usepackage[margin=1in]{geometry}
\usepackage[titletoc,title]{appendix}

\usepackage{hyperref}

% Math
% https://www.overleaf.com/learn/latex/Mathematical_expressions
% https://en.wikibooks.org/wiki/LaTeX/Mathematics
\usepackage{amsmath,amsfonts,amssymb,mathtools}

% Images
% https://www.overleaf.com/learn/latex/Inserting_Images
% https://en.wikibooks.org/wiki/LaTeX/Floats,_Figures_and_Captions
\usepackage{graphicx,float}

% Tables
% https://www.overleaf.com/learn/latex/Tables
% https://en.wikibooks.org/wiki/LaTeX/Tables

% Algorithms
% https://www.overleaf.com/learn/latex/algorithms
% https://en.wikibooks.org/wiki/LaTeX/Algorithms
\usepackage[ruled,vlined]{algorithm2e}
\usepackage{algorithmic}

% Code syntax highlighting
% https://www.overleaf.com/learn/latex/Code_Highlighting_with_minted
%\usepackage{minted}
%\usemintedstyle{borland}



% Title content
\title{Kalman Filter Application}
\author{Alssandro Riccardi}
\date{\today}

\begin{document}
    
        \maketitle
            
                % Abstract
                    \begin{abstract}
                            This is a simple exercise where you will have to complete the Kalman Filter algorithm.
                                \end{abstract}
                                    
                                        % Introduction and Overview
                                            \section{The Exercise}
                                                What you have to do is simple, implement the \textit{Propagation} and \textit{Correction} step of the Kalman Filter. I provide you with a framework where the 80\% of the work is already done, but still not complete. Find the correct functions that needs to be completed and complete them. Just remember, the required matrices are already implemented. 
                                                    
                                                        % Example Subsection
                                                            \subsection{The Software}
                                                                The \textit{kalman\_filter\_applied} is the name of the software. Its function is to simulate a trajectory and generate GPS measurements that will be recorded. With these recordings you can then estimate the truth trajectory by using the KF (that you have to complete).
                                                                    
                                                                        
                                                                            % Example Subsection
                                                                                \subsubsection{The Kalman Filter}
                                                                                    The Kalman Filter is an algorithm that uses a series of measurements observed over time, containing statistical noise and other inaccuracies, and produces estimates of unknown variables that tend to be more accurate than those based on a single measurement alone, by estimating a joint probability distribution over the variables for each timeframe. The filter is named after Rudolf E. Kálmán, one of the primary developers of its theory. 
                                                                                        
                                                                                            Some useful resources:
                                                                                                \begin{itemize}
                                                                                                        \item \url{https://en.wikipedia.org/wiki/Kalman_filter} 
                                                                                                                \item  \url{https://www.kalmanfilter.net/default.aspx}  
                                                                                                                    \end{itemize}
                                                                                                                        
                                                                                                                            
                                                                                                                                
                                                                                                                                \end{document}

